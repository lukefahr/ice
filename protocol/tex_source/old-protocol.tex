\documentclass{article}
\usepackage[layout=letterpaper,margin=1in]{geometry}

\usepackage{fancyhdr}
\usepackage{xspace}
\usepackage{color}

\usepackage[colorlinks=true]{hyperref}

\usepackage[all=normal,lists]{savetrees}

\usepackage[endianness=big]{bytefield}
\bytefieldsetup{boxformatting={\centering\footnotesize}}

\newcommand{\colorbitbox}[3]{%
\rlap{\bitbox{#2}{\color{#1}\rule{\width}{\height}}}%
\bitbox{#2}{#3}}
\definecolor{lightblue}{RGB}{0,204,255}
\definecolor{lightcyan}{rgb}{0.84,1,1}
\definecolor{lightgreen}{rgb}{0.64,1,0.71}
\definecolor{lightergreen}{rgb}{0.84,1,0.87}
\definecolor{lightred}{rgb}{1,0.7,0.71}

\usepackage{enumitem}
\setlistdepth{8}
%\setlist[itemize,1]{label=$\bullet$}
\setlist[itemize,1]{label=}
\setlist[itemize,2]{label=--}
\setlist[itemize,3]{label=$\circ$}
\setlist[itemize,4]{label=$\rightarrow$}
\setlist[itemize,5]{label=$\diamondsuit$}
\setlist[itemize,6]{label=$\cdot$}
\setlist[itemize,7]{label=$\star$}
\setlist[itemize,8]{label=$\ast$}
\renewlist{itemize}{itemize}{8}

\begin{document}

\fancyfoot[L]{Revision 0.1 -- Nov 29, 2012}
\pagestyle{fancyplain}

\title{ICE Debug Board FPGA \textcolor{red}{\bf OLD} Interface protocol}
\author{Ben Kempke, Pat Pannuto \{bpkempke,ppannuto\}@umich.edu}
\date{\color{red}{\em Note: This protocol applies only to the version 1 board
and is deprecated.}}
\maketitle

The ICE board is a dual-purpose development board, enabling both high-speed
programming (via DMA) of M3 chips and interfacing with live M3 sessions and
external peripherals.

This first-cut protocol only allows for basic I2C communication. Messages are
composed of three pieces, first a message type identifier (`c' or `d'), then
an arbitrary amount of message data, lastly a terminating CRLF sequence
(`$\backslash$r$\backslash$n')

\bigskip

\noindent
\begin{bytefield}{48}
\colorbitbox{lightgreen}{8}{Message Type} &
\bitbox{8}{Data...}
\bitbox{8}{Data...}
\bitbox{8}{Data...}
\colorbitbox{lightred}{8}{`$\backslash$r'} &
\colorbitbox{lightred}{8}{`$\backslash$n'} &
\end{bytefield}

\section{`d' Message}

The `d' message type is used to instruct the ICE board and tell it which I2C
addresses it should pretend to be. The entire ($2^{7} = 128$) I2C address
space is mapped to a bit mask. The ICE interface will ACK bytes sent to any
address with a bit set in the address mask (for either read or write requests). As example, the following message would configure ICE to respond to I2C addresses {\tt 0x00}, {\tt 0x02}, and
{\tt 0x98}.

\begin{verbatim}
Address Index:   00112233445566778899AABBCCDDEEFF
Command Message: 03000000000000000010000000000000

Complete Message: >>>dC0000000000000000010000000000000\r\n<<<
\end{verbatim}

There is no response from this message or any immediate indication that it did
anything.

\section{`c' Message}

The `c' message type is used to communicate I2C data. After the leading `c',
the subsequent bytes are the I2C data, sent byte for byte in order. Whenever
{\em any} bytes are sent over the I2C bus, they will be echoed back with a `c'
message. After the stop bit is seen on the bus, the terminating
$\backslash{r}\backslash{n}$ sequence is sent.

The I2C data is ASCII-encoded as capital hex characters. As example, the byte
{\tt 0b10110110} would be encoded as {B6}, this is Most-Significant-Half-Word
(MSHW) first and Least-Significant-Half-Word (LSHW) second.

\bigskip

\noindent
\begin{bytefield}{60}
\colorbitbox{lightgreen}{8}{`c'} &
\bitbox{8}{Address MSHW}
\bitbox{8}{Address LSHW}
\bitbox{8}{Data0 MSHW}
\bitbox{8}{Data0 LSHW}
\bitbox{4}{[...]}
\colorbitbox{lightred}{8}{`$\backslash$r'} &
\colorbitbox{lightred}{8}{`$\backslash$n'} &
\end{bytefield}


%%%%%%%%%%%%%%%%%%%%%%%%%%%%%%%%%%%%%%%%%%%%%%%%%%%%%%%
\pagebreak
\section{Document Revision History}
\label{sec:revisions}

\begin{itemize}

\item Revision 0.1 -- Nov 29, 2012
\subitem Initial revision

\end{itemize}

\end{document}
