\documentclass{article}
\usepackage[layout=letterpaper,margin=1in]{geometry}

\usepackage{fancyhdr}
\usepackage{xspace}
\usepackage{color}
\usepackage{comment}

\usepackage[colorlinks=true]{hyperref}

\usepackage[all=normal,lists]{savetrees}

\usepackage[endianness=big]{bytefield}
\bytefieldsetup{boxformatting={\centering\footnotesize}}

\usepackage{etoc}
\setcounter{tocdepth}{4}

\newcommand{\colorbitbox}[3]{%
\rlap{\bitbox{#2}{\color{#1}\rule{\width}{\height}}}%
\bitbox{#2}{#3}}
\definecolor{lightblue}{RGB}{0,204,255}
\definecolor{lightcyan}{rgb}{0.84,1,1}
\definecolor{lightgreen}{rgb}{0.64,1,0.71}
\definecolor{lightergreen}{rgb}{0.84,1,0.87}
\definecolor{lightred}{rgb}{1,0.7,0.71}

\usepackage{enumitem}
\setlistdepth{8}
%\setlist[itemize,1]{label=$\bullet$}
\setlist[itemize,1]{label=}
\setlist[itemize,2]{label=--}
\setlist[itemize,3]{label=$\circ$}
\setlist[itemize,4]{label=$\rightarrow$}
\setlist[itemize,5]{label=$\diamondsuit$}
\setlist[itemize,6]{label=$\cdot$}
\setlist[itemize,7]{label=$\star$}
\setlist[itemize,8]{label=$\ast$}
\renewlist{itemize}{itemize}{8}

\begin{document}

\fancyfoot[L]{Revision 0.2.6 -- Oct 17, 2013}
\pagestyle{fancyplain}

\title{ICE Debug Board FPGA Interface protocol}
\author{Ben Kempke, Pat Pannuto \{bpkempke,ppannuto\}@umich.edu}
\date{Revision 0.2.6 --- October 17, 2013}
\maketitle

\section*{Introduction}
The ICE board is a dual-purpose development board, enabling both high-speed
programming (via DMA) of M3 chips and interfacing with live M3 sessions and
external peripherals.

This means that libraries must be able to gracefully handle asychronous,
unexpected events. In particular, some semantics resemebling transactions
would be desirable, as well as an effort to maintain event ordering over the
serial communications link. The consequence is that this document should be
viewed less as a protocol specification and more of a living document as we
explore the best semantics for this domain.

At a high-level, every message is a composition of an {\em event id}, a
{\em message type} and an optional {\em message}. Every message sent {\em to}
the ICE board {\bf MUST} be replied to by either an ACK or NAK response.
Every message sent {\em from} the ICE board {\bf MUST
NOT} be acknowledged. The rationale for this asymmetry is to simplify the ICE
hardware design. With this mechanism, the hardware is not obligated to
preserve any communication state beyond the immediate running context. The
event ids provide a total ordering of the event history, insofar as is
possible.

\setcounter{tocdepth}{2}
%http://tex.stackexchange.com/questions/87709/minitoc-of-a-subsection/87716
%\etocsettocstyle{\subsection*{\contentsname}\hrule\medskip
%\begin{minipage}{.95\linewidth}}
%{\end{minipage}\medskip\hrule}
%\etocsettocstyle
%    {\subsection*{\contentsname}\hrule\medskip
%        \everypar{\rightskip.1\linewidth}}
%    {\nobreak\medskip\hrule\bigskip}
\tableofcontents

\clearpage
\section{Event Ids}
\label{sec:ids}

\subsection{Motivation}
Some decisions in microcontrollers necessarily race. As contrived example, if
two GPIO pins are defined as interrupts but are electrically connected in the
external circuit, and then line is pulled high, which interrupt fires first?
While the decision is arbitrary, there is motivation to define an ordering of
events in the system. The events observed by ICE can be replayed in the
M-ultaor for debugging, but the is only possible if they can be accurately
re-created.

In practice, this diverges to two ideas:

\subsubsection{Concurrent Events}
Two events could be labeled with the same event id, indicating that they
occurred too close together in time for the ICE to distinguish them. This
would require divergence in the simulator for debugging runs; do-able, but not
trivial.

\subsubsection{I/O Pass-Through}
\label{sec:pass-through}
Currently, the FPGA and the M3 share GPIO pins (Nx2 header):
\begin{verbatim}
          /-----\
FPGA ---- | o o | ----- M3
          \-----/
          GPIO N
\end{verbatim}
Instead perhaps we can explicitly pass all I/O through the FPGA:
\begin{verbatim}
/---\
| o | ---- FPGA -- {TP} -- M3
\---/
GPIO N
\end{verbatim}
This has the disadvantage of doubling the number of FPGA I/O pins consumed for
each of the M3 I/O pins. It does enable the FPGA to strictly define an order
of events. We can possibly explore this with the current board by mapping
GPIOs 16-23 as FPGA-input only, jumpering over GPIOs 8-15 and having the FPGA
drive those, leaving GPIOs 0-7 as the first one.

\subsection{Implementation}

Event IDs are an unsigned single byte number. They define a total ordering the
events actually occurred in the system. In particular, if a command message is
being sent to set GPIO 0 (an output) high at the same time that GPIO 1 (an
input) goes high, the ordering will be {\tt Event N: GPIO 1 --> High} then
{\tt Event N+1: GPIO 0 --> High}. With \ref{sec:pass-through}'s pass through,
perhaps the GPIO 1 transition could be delayed during command message
reception, but this is for further exploration.

For message format consistency, event ids are included in both directions of
communication, however the field may be safely ignored by the FPGA. The FPGA
itself must by definition have some order of I/O events it processed, which
are encapsulated by these event ids. The event id of a control message is
assigned by the FPGA whenever it actually processes the event and is indicated
to the controller via the ACK message.

\clearpage
\section{Messages / Base Protocol}
\label{sec:messages}

Messages are composed of a
one-byte message type identifier
followed by a one-byte event identifier
followed by a one-byte unsigned length indicator
followed by an optional message component. Visually:
\begin{quote}
\begin{bytefield}{32} \\
\colorbitbox{lightgreen}{8}{Message Type} &
\colorbitbox{lightred}{8}{Event ID} &
\colorbitbox{lightcyan}{8}{Length} &
\bitbox{8}{[Message Data...]} &
\end{bytefield}
\end{quote}
Effort
should be made to keep type identifiers within the ASCII range where
reasonable, mapped to appropriate letters.

There are two types of transactions defined: {\em synchronous} and {\em
asynchronous}. A synchronous message is one initiated by the controlling PC
and must be responded to by a \{N\}ACK from the ICE board. An asychronous
message is generated by the ICE board in response to a hardware event. Only
one synchronous message is permitted to be live at any given time. The ICE
board may send any number of asychronous messages before responding to the
synchronous message.

The protocol in use is undefined
until a version request is ACKed. The reception of a {\em NAK with reason}
with a preferred version is {\bf NOT} sufficient to establish a version, the
controller {\bf MUST} explicilty send another {\bf v}ersion request and
receive an ACK to establish the protocol version in use.

{\bf Only `V' and `v' messages are valid until a version has been negotiated.}
Any asynchronous hardware events may either be queued or discarded.

\medskip

The base protocol defines the following immutable message types and their
properties:

\begin{itemize}

\item {\textbf {\texttt 0x00} ACK}
\item {\textbf {\texttt 0x01} NAK}
\item {\em Synchronous Response}
\begin{itemize}
	\item[]
\begin{bytefield}{32} \\
\colorbitbox{lightgreen}{8}{0x00 or 0x01} &
\colorbitbox{lightred}{8}{Event ID} &
\colorbitbox{lightcyan}{8}{Len (Min: 0)} &
\bitbox{8}{[Data...]}
\end{bytefield}
	\begin{itemize}
		\item {\bf 0:} ACK. Indicates success.
		\item {\bf 1:} NAK. Generic error code indicating failure.
	\end{itemize}
	\item Unless otherwise specified...
	\begin{itemize}
		\item The remaining bytes shall be an ASCII-encoded string
		composed only of standard printable characters. The string
		shall not be NUL-terminated. The contents of this string is
		not well-defined and is expected to be something
		human-readable and useful.
	\end{itemize}
	\item If specified...
	\begin{itemize}
		\item The response is permitted to be implementation defined.
	\end{itemize}
\end{itemize}

\item \textbf{\texttt{0x56 (`V')} -- Query Versions}
\item {\em Synchronous Request}
\begin{itemize}
	\item[]
\begin{bytefield}{24} \\
\colorbitbox{lightgreen}{8}{0x56} &
\colorbitbox{lightred}{8}{Event ID} &
\colorbitbox{lightcyan}{8}{Len (Must be 0)} &
\end{bytefield}
	\item This message queries the protocol version(s) this ICE
implementation understands.
	\item If multiple versions are supported, they should be listed in
order of version preference.
	\item Response:

		\begin{bytefield}{56} \\
		\colorbitbox{lightgreen}{8}{0x00} &
		\colorbitbox{lightred}{8}{Event ID} &
		\colorbitbox{lightcyan}{8}{Len (Min: 2)} &
		\bitbox{8}{Major} &
		\bitbox{8}{Minor} &
		\bitbox{8}{[Major...]} &
		\bitbox{8}{[Minor...]} &
		\end{bytefield}
\end{itemize}

\item \textbf{\texttt{0x76 (`v')} -- Request to use version}
\item {\em Synchronous Request}
\begin{itemize}
	\item[]
\begin{bytefield}{40} \\
\colorbitbox{lightgreen}{8}{0x76} &
\colorbitbox{lightred}{8}{Event ID} &
\colorbitbox{lightcyan}{8}{Len (Must be 2)} &
\bitbox{8}{Major} & \bitbox{8}{Minor}
\end{bytefield}
	\item The message shall be composed of exactly two bytes.
	\item Each byte shall be an unsigned number.
	\item The first byte shall be considered a major version number.
	\item The second byte shall be considered a minor version number.
	\item Response:
	\begin{itemize}
		\item[]
\begin{bytefield}{24} \\
\colorbitbox{lightgreen}{8}{0x00} &
\colorbitbox{lightred}{8}{Event ID} &
\colorbitbox{lightcyan}{8}{Len (Must be 0)}
\end{bytefield}
		\item If the selected version is acceptable, an ACK shall be generated.
		\item[]
\begin{bytefield}{40} \\
\colorbitbox{lightgreen}{8}{0x01} &
\colorbitbox{lightred}{8}{Event ID} &
\colorbitbox{lightcyan}{8}{Length} &
\bitbox{8}{[Major]} & \bitbox{8}{[Minor]} &
\end{bytefield}
		\item Otherwise a NAK shall be generated.
		\item The NAK may optionally include a preferred version or
list of versions in the same format as the `V' response.
	\end{itemize}
	\item This message is not queriable.
\end{itemize}

\item \textbf{\texttt{0x58 (`X')} -- eXtension}
\item \textbf{\texttt{0x78 (`x')} -- extension}
\begin{itemize}
\item These characters are reserved for future e{\bf X}tentions.
\item No further specification is defined here.
\end{itemize}

\end{itemize}

%%%%%%%%%%%%%%%%%%%%%%%%%%%%%%%%%%%%%%%%%%%%%%%%%%%%%%%
\pagebreak
\section{Protocols}
\label{sec:protocols}

The following protocols are currently well-defined:
\setcounter{tocdepth}{2}
%\etocsettocstyle{\subsection*{Versions}\hrule\medskip
%\begin{minipage}{.95\linewidth}}
%{\end{minipage}\medskip\hrule}
\etocsettocstyle
    {\subsection*{\contentsname}\hrule\medskip
        \everypar{\rightskip.1\linewidth}}
    {\nobreak\medskip\hrule\bigskip}
\localtableofcontents

\clearpage

\subsection{Version 0.1}
\label{protocol-0-1}

\setcounter{tocdepth}{4}
\etocsettocstyle
    {\subsection*{\contentsname}\hrule\medskip
        \everypar{\rightskip.1\linewidth}}
    {\nobreak\medskip\hrule\bigskip}

\localtableofcontents

\subsubsection{\texttt{0x64 `d'} -- Discrete interface I2C message}
{\em Synchronous Request, Asynchronous Message}
\begin{itemize}
  \item[]
    \begin{bytefield}{40}
      \colorbitbox{lightgreen}{8}{`d'} &
      \colorbitbox{lightred}{8}{Event ID} &
      \colorbitbox{lightcyan}{8}{Length} &
      \bitbox{4}{Data...}
    \end{bytefield}
  \item The bytes in this message compose an I2C transaction.
  \item The length field of this message necessarily limits the maximum
    message size to 255 bytes (addr + 254 data). Longer messages should
    be fragmented.
  \item The sentinel value {\tt 255} for length indicates a {\em
    fragmented} message.
  \item[]
    \begin{itemize}
      \item A message fragment {\bf MUST} be exactly 255 bytes long.
      \item A fragment message {\bf MUST} be followed by another
        fragment, or terminated by a regular {\bf d}iscrete message.
        \begin{itemize}
          \item The ICE board is permitted to interleave other,
            non-I2C related messages (e.g. GPIO events).
        \end{itemize}
      \item A series of message fragments {\bf MUST} always be
        terminated by a discrete with an explicit length. An I2C transaction of
        length 0 is permitted, e.g.:
        \begin{itemize}
          \item An I2C transaction of length 510 (1 byte addr + 509 bytes
            of data) would be {\bf three} messages. The first of length
            255 (addr + bytes 0-253), the second of length 255 (bytes
            254-508), and the third of length 0 (there is no more
            data, but the fragment series must be terminated).
        \end{itemize}
      \item Fragments are treated as one logical message, but individual I2C bus
        transactions, by an ICE board.  In practice this means:
        \begin{itemize}
          \item Each fragment message must be individually ACK'd by ICE.
            \begin{itemize}
              \item A NAK'd fragment message ends an I2C message.
              \item The NAK offset is relative to the current fragment, not
                the whole I2C transaction.
            \end{itemize}
          \item Only the first fragment includes the I2C address.
          \item A stop bit should {\bf NOT} be generated after a
            fragment, instead the I2C clock should be stretched until the
            next fragment has arrived.
        \end{itemize}
    \end{itemize}
  \item ICE will respond with an ACK once every byte from an individual `d'
    message has been ACK'd on the I2C bus.
  \item If a byte is NAK'd on the I2C bus, ICE will respond with a NAK message
    of length 1 indicating the index of the first NAK'd byte (e.g. if the
    address is NAK'd, it will return 0).
    \begin{itemize}
      \item[]
        \begin{bytefield}{32}
          \colorbitbox{lightgreen}{8}{NAK (0x01)} &
          \colorbitbox{lightred}{8}{Event ID} &
          \colorbitbox{lightcyan}{8}{Len (Must be 1)} &
          \bitbox{8}{Index of Byte NAK'd}
        \end{bytefield}
    \end{itemize}
\end{itemize}

\subsubsection{\texttt{0x49 `I'} -- Query I2C Configuration}
{\em Synchronous Request}
\begin{itemize}
  \item[]
    \begin{bytefield} \\
      \colorbitbox{lightgreen}{8}{0x49} &
      \colorbitbox{lightred}{8}{Event ID} &
      \colorbitbox{lightcyan}{8}{Len (Must be 1)} &
      \colorbitbox{lightergreen}{8}{Parameter} &
    \end{bytefield}
  \item These messages complement the set I2C messages.
  \item The `Parameter' field is the parameter specifier to query.
  \item An ACK response should mimic the corresponding set message.
  \item The following would query/response the address mask:
    \begin{quote}
      \begin{bytefield} \\
        \colorbitbox{lightgreen}{8}{0x49} &
        \colorbitbox{lightred}{8}{Event ID} &
        \colorbitbox{lightcyan}{8}{0x01} &
        \colorbitbox{lightergreen}{8}{0x61} &
      \end{bytefield}

      \begin{bytefield} \\
        \colorbitbox{lightgreen}{8}{ACK (0x00)} &
        \colorbitbox{lightred}{8}{Event ID} &
        \colorbitbox{lightcyan}{8}{0x02} &
        \bitbox{8}{Ones Mask} &
        \bitbox{8}{Zeros Mask} &
      \end{bytefield}
    \end{quote}
\end{itemize}

\subsubsection{\texttt{0x69 `i'} -- Set I2C Configuration}
{\em Synchronous Request}
\begin{itemize}
  \item The first byte of the message shall define which parameter is to
    be configured.

  \paragraph{\texttt{0x69 0x63 `ic'} -- Set I2C Clock Speed}
    \begin{itemize}
      \item[]
        \begin{bytefield} \\
          \colorbitbox{lightgreen}{8}{`i'} &
          \colorbitbox{lightred}{8}{Event ID} &
          \colorbitbox{lightcyan}{8}{Len (Must be 2)} &
          \colorbitbox{lightergreen}{8}{`c'} &
          \bitbox{8}{Clock Speed}
        \end{bytefield}
      \item {\bf Default:} {\tt 0x32} (50, 100~kHz)
      \item This shall be followed by a single byte
        valued N, where N~*~2~kHz yeilds the desired clock speed. Values of N
        greater than 200 (400~kHz) exceed the I2C spec and may be rejected.
    \end{itemize}
  \paragraph{\texttt{0x69 0x61 `ia'} -- Set ICE I2C Address}
    \begin{itemize}
      \item[]
        \begin{bytefield} \\
          \colorbitbox{lightgreen}{8}{`i'} &
          \colorbitbox{lightred}{8}{Event ID} &
          \colorbitbox{lightcyan}{8}{Len (Must be 3)} &
          \colorbitbox{lightergreen}{8}{`a'} &
          \bitbox{8}{Ones Mask} &
          \bitbox{8}{Zeros Mask}
        \end{bytefield}
      \item {\bf Default:} {\tt 0xff 0xff} (disabled)
      \item This shall be followed by two bytes,
        first the {\em ones mask} and then the {\em zeroes mask} as outlined below.
        The command sets the address mask that ICE board should pretend to be a device
        for. Coneceptually the mask is of the form {\tt 10xx010x}, where x's signify
        don't care. This is conveyed as a {\em ones mask} and a {\em zeroes mask},
        where each mask defines the bits that must be a one or zero respectively. For
        the given example, the ones mask would be {\tt 10000100} and the zeroes mask
        {\tt 01001010}, generating a transaction of \mbox{\tt 0x61 0x84 0x4a}.
      \item To disable address-faking, set any bit
        as both required-one and required-zero. This impossible situation is a legal
        setting that will never match.
        \begin{itemize}
          \item {\em Note:} While it is
            permissable to set the last bit must-be-zero (writeable-only) or must-be-one
            (readable-only), it is almost certainly an error to do so.
        \end{itemize}
    \end{itemize}
  \paragraph{\texttt{0x69} Responses}
    \begin{itemize}
      \item NAKs for this message shall be composed of an error
        code, followed by an optional explanitory string.
        \begin{itemize}
          \item[]
            \begin{bytefield} \\
              \colorbitbox{lightgreen}{8}{NAK (0x01)} &
              \colorbitbox{lightred}{8}{Event ID} &
              \colorbitbox{lightcyan}{8}{Len (Min: 1)} &
              \colorbitbox{lightblue}{8}{EINVAL (0x16)} &
              \bitbox{8}{[{\tt "Out of Range"}]} &
            \end{bytefield}
          \item {\texttt {\textbf EINVAL (22,0x16):}} Invalid argument.
          \item[]
            \begin{bytefield} \\
              \colorbitbox{lightgreen}{8}{NAK (0x01)} &
              \colorbitbox{lightred}{8}{Event ID} &
              \colorbitbox{lightcyan}{8}{Len (Min: 1)} &
              \colorbitbox{lightblue}{8}{ENODEV (0x13)} &
            \end{bytefield}
          \item {\texttt {\textbf ENODEV (19,0x13):}} The
            implementation does not support changing or querying this parameter. Unless
            otherwise specified, it {\bf MUST} be hardcoded to the default.
        \end{itemize}
    \end{itemize}
\end{itemize}

\subsubsection{\texttt{0x66 `f'} -- FLOW (GOC) interface message}
{\em Synchronous Request}
\begin{itemize}
  \item FLOW messages are formatted the exact same as `d'iscrete messages.
\end{itemize}

\subsubsection{\texttt{0x4f `O'} -- Query optical (FLOW (GOC)) Configuration}
{\em Synchronous Request}
\begin{itemize}
  \item[]
    \begin{bytefield} \\
      \colorbitbox{lightgreen}{8}{`O' (0x4f)} &
      \colorbitbox{lightred}{8}{Event ID} &
      \colorbitbox{lightcyan}{8}{Len (Must be 1)} &
      \colorbitbox{lightergreen}{8}{Parameter} &
    \end{bytefield}
  \item These messages complement the set I2C messages.
  \item The `Parameter' field is the parameter specifier to query.
  \item An ACK response should mimic the corresponding set message.
\end{itemize}

\subsubsection{\texttt{0x6f `o'} -- Set optical (FLOW (GOC)) Configuration}
{\em Synchronous Request}
\begin{itemize}
  \item The first byte of the message shall define which parameter is to
    be configured.
    \paragraph{\texttt{0x6f 0x63 `oc'}: Clock Speed (Divider)}
      \begin{itemize}
        \item[]
          \begin{bytefield} \\
            \colorbitbox{lightgreen}{8}{`o'} &
            \colorbitbox{lightred}{8}{Event ID} &
            \colorbitbox{lightcyan}{8}{Len (Must be 4)} &
            \colorbitbox{lightergreen}{8}{`c'} &
            \bitbox{8}{Clock Divider}
          \end{bytefield}
        \item {\bf Default:} {\tt 0x30D400} (2~MHz / 0x30D400 = .625~Hz)
        \item This shall be followed by a three byte
          value N (MSB-first), where 2~MHz / N yields the desired clock speed.
      \end{itemize}
  \paragraph{\texttt{0x6f} Responses}
    \begin{itemize}
      \item NAKs for this message shall be composed of an error
        code, followed by an optional explanitory string.
        \begin{itemize}
          \item[]
            \begin{bytefield} \\
              \colorbitbox{lightgreen}{8}{NAK (0x01)} &
              \colorbitbox{lightred}{8}{Event ID} &
              \colorbitbox{lightcyan}{8}{Len (Min: 1)} &
              \colorbitbox{lightblue}{8}{EINVAL (0x16)} &
              \bitbox{8}{[{\tt "Out of Range"}]} &
            \end{bytefield}
          \item {\texttt {\textbf EINVAL (22,0x16):}} Invalid argument.
        \end{itemize}
    \end{itemize}
\end{itemize}

\subsubsection{\texttt{0x47 `G'} -- Query GPIO State / Configuration}
{\em Synchronous Request}
\begin{itemize}
  \item[]
    \begin{bytefield} \\
      \colorbitbox{lightgreen}{8}{`G' (0x47)} &
      \colorbitbox{lightred}{8}{Event ID} &
      \colorbitbox{lightcyan}{8}{Len (Must be 2)} &
      \colorbitbox{lightergreen}{8}{Parameter} &
      \bitbox{8}{GPIO IDX}
    \end{bytefield}
  \item These messages complement the set GPIO (`g') messages.
  \item The `Parameter' field is the parameter specifier to query.
  \item An ACK response should mimic the corresponding set message.
\end{itemize}

\subsubsection{\texttt{0x67 `g'} -- Set / Configure GPIO}
{\em Synchronous Request, Asynchronous Message}
\begin{itemize}
  \item The first byte of this message shall be a specificer, indicating
    what type of GPIO action is requested.
  \paragraph{\texttt{0x67 0x6c `gl'} -- GPIO Level}
    \begin{itemize}
      \item The first byte of the message shall be an integer
        indicating the GPIO index to set. The second byte shall be
        valued {\tt 0} or {\tt 1}, depending on the desired GPIO
        state.
      \item[]
        \begin{bytefield} \\
          \colorbitbox{lightgreen}{8}{`o' (0x67)} &
          \colorbitbox{lightred}{8}{Event ID} &
          \colorbitbox{lightcyan}{8}{Len (Must be 3)} &
          \colorbitbox{lightergreen}{8}{`l' (0x6c)} &
          \bitbox{8}{GPIO IDX} &
          \bitbox{8}{GPIO Val}
        \end{bytefield}
    \end{itemize}
  \paragraph{\texttt{0x67 0x64 `gd'} -- GPIO direction}
    \begin{itemize}
      \item The first byte of the message shall be an integer
        indicating the GPIO index to set the direction of. The second
        byte shall be valued:
        \begin{itemize}
          \item 0: Input
          \item 1: Output
          \item 2: TriState (\textsc{DEFAULT})
        \end{itemize}
      \item[]
        \begin{bytefield} \\
          \colorbitbox{lightgreen}{8}{`o' (0x67)} &
          \colorbitbox{lightred}{8}{Event ID} &
          \colorbitbox{lightcyan}{8}{Len (Must be 3))} &
          \colorbitbox{lightergreen}{8}{`d' (0x64)} &
          \bitbox{8}{GPIO IDX} &
          \bitbox{8}{GPIO Direction}
        \end{bytefield}
    \end{itemize}
  \paragraph{\texttt{0x67} Responses}
    \begin{itemize}
      \item {\texttt {\textbf ENODEV (19,0x13):}} The requested
        GPIO does not exist.
      \item[]
        \begin{bytefield} \\
          \colorbitbox{lightgreen}{8}{NAK (0x01)} &
          \colorbitbox{lightred}{8}{Event ID} &
          \colorbitbox{lightcyan}{8}{Len (Min: 1)} &
          \colorbitbox{lightblue}{8}{ENODEV (0x13)} &
          \bitbox{8}{[{\tt "No such GPIO"}]} &
        \end{bytefield}
      \item {\texttt {\textbf EINVAL (22,0x16):}} The requested GPIO
        exists, but cannot be configured this way at this time.
      \item[]
        \begin{bytefield} \\
          \colorbitbox{lightgreen}{8}{NAK (0x01)} &
          \colorbitbox{lightred}{8}{Event ID} &
          \colorbitbox{lightcyan}{8}{Len (Min: 1)} &
          \colorbitbox{lightblue}{8}{EINVAL (0x16)} &
          \bitbox{8}{[{\tt "GPIO is input"}]} &
        \end{bytefield}
    \end{itemize}
\end{itemize}

\subsubsection{\texttt{0x50 `P'} -- Query Power State}
{\em Synchronous Request}
\begin{itemize}
  \item[]
    \begin{bytefield} \\
      \colorbitbox{lightgreen}{8}{`P' (0x50)} &
      \colorbitbox{lightred}{8}{Event ID} &
      \colorbitbox{lightcyan}{8}{Len (Must be 1)} &
      \colorbitbox{lightergreen}{8}{Parameter} &
      \bitbox{8}{PWR IDX}
    \end{bytefield}
  \item These messages complement the Set Power (`p') messages.
  \item The `Parameter' field is the parameter specifier to query.
  \item An ACK response should mimic the corresponding set message.
\end{itemize}

\subsubsection{\texttt{0x70 `p'} -- Set Power State}
{\em Synchronous Request, Asynchronous Message}
\begin{itemize}
  \item Set Power State messages allow direct control of set-point voltage and on/off states for various power domains on the ICE board.  The first byte of this message shall be a specificer, indicating which parameter is requested.  The second byte of the message shall be the power domain identifier.  Currently implemented power domain identifiers are:
   \begin{itemize}
     \item 0: M3 0.6V (0.675V Default)
     \item 1: M3 1.2V (1.2V Default)
     \item 2: M3 VBatt (3.8V Default)
   \end{itemize}
  \paragraph{\texttt{0x70 0x76 `pv'} -- Voltage State}
    \begin{itemize}
      \item The first byte of the message shall be a single byte indicating the power domain identifier to set.  The second byte ($v\_set$) shall specify the voltage according to the equation:
$$V_{out} = (0.537 + 0.0185*v\_set)*V_{default}$$
      Valid values for $v\_set$ range from {\tt 0} to {\tt 31}
      \item[]
        \begin{bytefield} \\
          \colorbitbox{lightgreen}{8}{`p' (0x70)} &
          \colorbitbox{lightred}{8}{Event ID} &
          \colorbitbox{lightcyan}{8}{Len (Must be 3)} &
          \colorbitbox{lightergreen}{8}{`v' (0x76)} &
          \bitbox{8}{PWR IDX} &
          \bitbox{8}{$v\_set$}
        \end{bytefield}
    \end{itemize}
  \paragraph{\texttt{0x70 0x6f `po'} -- On/Off State}
    \begin{itemize}
      \item The first byte of the message shall be a single byte indicating the power domain identifier to set.  The second byte shall be
        valued {\tt 0} or {\tt 1}, depending on the desired On/Off
        state.
      \item[]
        \begin{bytefield} \\
          \colorbitbox{lightgreen}{8}{`p' (0x70)} &
          \colorbitbox{lightred}{8}{Event ID} &
          \colorbitbox{lightcyan}{8}{Len (Must be 3)} &
          \colorbitbox{lightergreen}{8}{`o' (0x6f)} &
          \bitbox{8}{PWR IDX} &
          \bitbox{8}{On/Off}
        \end{bytefield}
    \end{itemize}
\end{itemize}

\clearpage
\subsection{Version 0.2}
\label{protocol-0-2}

Version 0.2 adds the following new messages:
\begin{itemize}
  \item {\tt `Oo'} and {\tt `oo'}. These messages allow the default on/off state of
the FLOW light to be set.
  \item {\tt `??'}. This message queries the capabilities of the ICE board.
  \item {\tt `?B'}, {\tt `?b'}. These messages get and set the ICE baud rate.
  \item {\tt `B'}, {\tt `b'}, {\tt `M'}, and {\tt `m'}. These messages control
    the MBus interface.
  \item {\tt `e'}. This command injects messages on the {\tt EIN} debug port.
\end{itemize}

\setcounter{tocdepth}{4}
\etocsettocstyle
    {\subsection*{\contentsname}\hrule\medskip
        \everypar{\rightskip.1\linewidth}}
    {\nobreak\medskip\hrule\bigskip}

\localtableofcontents

\subsubsection{\texttt{0x3f `?'} -- Query ICE}
{\em Synchronous Request}
\begin{itemize}
  \item The first byte of this message shall be a specificer, indicating
    what type of query is requested.
  \paragraph{\texttt{0x3f 0x3f `??'} -- Query capabilities}
    \begin{itemize}
      \item This message shall query the capabilities of this ICE board. Not
        all boards have every hardware frontend. Use this message to query the
        capabilities of a given ICE board.
      \item[]
        \begin{bytefield} \\
          \colorbitbox{lightgreen}{8}{`?' (0x3f)} &
          \colorbitbox{lightred}{8}{Event ID} &
          \colorbitbox{lightcyan}{8}{Len (Must be 1)} &
          \colorbitbox{lightergreen}{8}{`?' (0x3f)}
        \end{bytefield}
      \item This message shall be responded to with a list of all top-level
        identifiers that the ICE board is capable of acting usefully upon.
        That is, if the ICE firmware understands {\tt `d'} messages but does
        not have an I2C frontend {\tt `d'} shall be omitted. Sub-types are not
        specified by this command. That is, if a version 0.2 ICE includes
        {\tt `o'} in its response it is assumed to understand both the
        {\tt `oc'} and {\tt `oo'} messages.
      \item Both set and query commands should be included in this list. That
        is, if a FLOW/GOC frontend is present, but {\tt `o'} and {\tt `O'}
        should be included.
      \item As example, a version 0.2 ICE with no physical I2C frontend would
        respond:
      \item[]
        \begin{bytefield} \\
          \colorbitbox{lightgreen}{8}{ACK (0x00)} &
          \colorbitbox{lightred}{8}{Event ID} &
          \colorbitbox{lightcyan}{8}{Len (10)} &
          \bitbox{20}{\texttt{`?IifOoGgPp' (0x3f4969664f6f47675070)}} &
        \end{bytefield}
    \end{itemize}
  \paragraph{\texttt{0x3f 0x42 `?B'} -- Query baudrate}
    \begin{itemize}
      \item This message shall mirror the {\tt `?b'} message and report the
        currently set baud rate.
      \item Or maybe it should report all the supported baud rates. If you're
        sending / receiving messages in the first place, you probably already
        know the baud rate.
    \end{itemize}
  \paragraph{\texttt{0x3f 0x62 `?b'} -- Set baudrate}
    \begin{itemize}
      \item This message shall set the baud rate for future messages. The new
        baud rate shall take effect \hl{immediately OR? after the ACK for this
        request}. (Probably should take effect after ACK'ing so that you can
        NACK an illegal / unsupported value?).
      \item The baud rate shall be specified as.... {\hl a divisor?}.
      \item \textbf{Default Value:} \hl{Whatever value corrolates to 115200.}
      \item The following message would set the ICE speed to 3 Megabaud:
      \item[]
        \begin{bytefield} \\
          \colorbitbox{lightgreen}{8}{`?' (0x3f)} &
          \colorbitbox{lightred}{8}{Event ID} &
          \colorbitbox{lightcyan}{8}{Length} &
          \colorbitbox{lightergreen}{8}{`b' (0x62)}
          \bitbox{8}{\hl{VALUE}}
        \end{bytefield}
    \end{itemize}
\end{itemize}

\subsubsection{\texttt{0x64 `d'} -- Discrete interface I2C message}
{\em Synchronous Request, Asynchronous Message}
\begin{itemize}
  \item[]
    \begin{bytefield}{40}
      \colorbitbox{lightgreen}{8}{`d'} &
      \colorbitbox{lightred}{8}{Event ID} &
      \colorbitbox{lightcyan}{8}{Length} &
      \bitbox{4}{Data...}
    \end{bytefield}
  \item The bytes in this message compose an I2C transaction.
  \item The length field of this message necessarily limits the maximum
    message size to 255 bytes (addr + 254 data). Longer messages should
    be fragmented.
  \item The sentinel value {\tt 255} for length indicates a {\em
    fragmented} message.
  \item[]
    \begin{itemize}
      \item A message fragment {\bf MUST} be exactly 255 bytes long.
      \item A fragment message {\bf MUST} be followed by another
        fragment, or terminated by a regular {\bf d}iscrete message.
        \begin{itemize}
          \item The ICE board is permitted to interleave other,
            non-I2C related messages (e.g. GPIO events).
        \end{itemize}
      \item A series of message fragments {\bf MUST} always be
        terminated by a discrete with an explicit length. An I2C transaction of
        length 0 is permitted, e.g.:
        \begin{itemize}
          \item An I2C transaction of length 510 (1 byte addr + 509 bytes
            of data) would be {\bf three} messages. The first of length
            255 (addr + bytes 0-253), the second of length 255 (bytes
            254-508), and the third of length 0 (there is no more
            data, but the fragment series must be terminated).
        \end{itemize}
      \item Fragments are treated as one logical message, but individual I2C bus
        transactions, by an ICE board.  In practice this means:
        \begin{itemize}
          \item Each fragment message must be individually ACK'd by ICE.
            \begin{itemize}
              \item A NAK'd fragment message ends an I2C message.
              \item The NAK offset is relative to the current fragment, not
                the whole I2C transaction.
            \end{itemize}
          \item Only the first fragment includes the I2C address.
          \item A stop bit should {\bf NOT} be generated after a
            fragment, instead the I2C clock should be stretched until the
            next fragment has arrived.
        \end{itemize}
    \end{itemize}
  \item ICE will respond with an ACK once every byte from an individual `d'
    message has been ACK'd on the I2C bus.
  \item If a byte is NAK'd on the I2C bus, ICE will respond with a NAK message
    of length 1 indicating the index of the first NAK'd byte (e.g. if the
    address is NAK'd, it will return 0).
    \begin{itemize}
      \item[]
        \begin{bytefield}{32}
          \colorbitbox{lightgreen}{8}{NAK (0x01)} &
          \colorbitbox{lightred}{8}{Event ID} &
          \colorbitbox{lightcyan}{8}{Len (Must be 1)} &
          \bitbox{8}{Index of Byte NAK'd}
        \end{bytefield}
    \end{itemize}
\end{itemize}

\subsubsection{\texttt{0x49 `I'} -- Query I2C Configuration}
{\em Synchronous Request}
\begin{itemize}
  \item[]
    \begin{bytefield} \\
      \colorbitbox{lightgreen}{8}{0x49} &
      \colorbitbox{lightred}{8}{Event ID} &
      \colorbitbox{lightcyan}{8}{Length} &
      \bitbox{8}{Parameter} &
    \end{bytefield}
  \item These messages complement the set I2C messages.
  \item The `Parameter' field is the parameter specifier to query.
  \item An ACK response should mimic the corresponding set message.
  \item The following would query/response the address mask:
    \begin{quote}
      \begin{bytefield} \\
        \colorbitbox{lightgreen}{8}{0x49} &
        \colorbitbox{lightred}{8}{Event ID} &
        \colorbitbox{lightcyan}{8}{0x01} &
        \colorbitbox{lightergreen}{8}{0x61} &
      \end{bytefield}

      \begin{bytefield} \\
        \colorbitbox{lightgreen}{8}{ACK (0x00)} &
        \colorbitbox{lightred}{8}{Event ID} &
        \colorbitbox{lightcyan}{8}{0x02} &
        \bitbox{8}{Ones Mask} &
        \bitbox{8}{Zeros Mask} &
      \end{bytefield}
    \end{quote}
\end{itemize}

\subsubsection{\texttt{0x69 `i'} -- Set I2C Configuration}
{\em Synchronous Request}
\begin{itemize}
  \item The first byte of the message shall define which parameter is to
    be configured.

  \paragraph{\texttt{0x69 0x63 `ic'} -- Set I2C Clock Speed}
    \begin{itemize}
      \item[]
        \begin{bytefield} \\
          \colorbitbox{lightgreen}{8}{`i'} &
          \colorbitbox{lightred}{8}{Event ID} &
          \colorbitbox{lightcyan}{8}{Len (Must be 2)} &
          \colorbitbox{lightergreen}{8}{`c'} &
          \bitbox{8}{Clock Speed}
        \end{bytefield}
      \item {\bf Default:} {\tt 0x32} (50, 100~kHz)
      \item This shall be followed by a single byte
        valued N, where N~*~2~kHz yeilds the desired clock speed. Values of N
        greater than 200 (400~kHz) exceed the I2C spec and may be rejected.
    \end{itemize}
  \paragraph{\texttt{0x69 0x61 `ia'} -- Set ICE I2C Address}
    \begin{itemize}
      \item[]
        \begin{bytefield} \\
          \colorbitbox{lightgreen}{8}{`i'} &
          \colorbitbox{lightred}{8}{Event ID} &
          \colorbitbox{lightcyan}{8}{Len (Must be 3)} &
          \colorbitbox{lightergreen}{8}{`a'} &
          \bitbox{8}{Ones Mask} &
          \bitbox{8}{Zeros Mask}
        \end{bytefield}
      \item {\bf Default:} {\tt 0xff 0xff} (disabled)
      \item This shall be followed by two bytes,
        first the {\em ones mask} and then the {\em zeroes mask} as outlined below.
        The command sets the address mask that ICE board should pretend to be a device
        for. Coneceptually the mask is of the form {\tt 10xx010x}, where x's signify
        don't care. This is conveyed as a {\em ones mask} and a {\em zeroes mask},
        where each mask defines the bits that must be a one or zero respectively. For
        the given example, the ones mask would be {\tt 10000100} and the zeroes mask
        {\tt 01001010}, generating a transaction of \mbox{\tt 0x61 0x84 0x4a}.
      \item To disable address-faking, set any bit
        as both required-one and required-zero. This impossible situation is a legal
        setting that will never match.
        \begin{itemize}
          \item {\em Note:} While it is
            permissable to set the last bit must-be-zero (writeable-only) or must-be-one
            (readable-only), it is almost certainly an error to do so.
        \end{itemize}
    \end{itemize}
  \paragraph{\texttt{0x69} Responses}
    \begin{itemize}
      \item NAKs for this message shall be composed of an error
        code, followed by an optional explanitory string.
        \begin{itemize}
          \item[]
            \begin{bytefield} \\
              \colorbitbox{lightgreen}{8}{NAK (0x01)} &
              \colorbitbox{lightred}{8}{Event ID} &
              \colorbitbox{lightcyan}{8}{Len (Min: 1)} &
              \colorbitbox{lightblue}{8}{EINVAL (0x16)} &
              \bitbox{8}{[{\tt "Out of Range"}]} &
            \end{bytefield}
          \item {\texttt {\textbf EINVAL (22,0x16):}} Invalid argument.
          \item[]
            \begin{bytefield} \\
              \colorbitbox{lightgreen}{8}{NAK (0x01)} &
              \colorbitbox{lightred}{8}{Event ID} &
              \colorbitbox{lightcyan}{8}{Len (Min: 1)} &
              \colorbitbox{lightblue}{8}{ENODEV (0x13)} &
            \end{bytefield}
          \item {\texttt {\textbf ENODEV (19,0x13):}} The
            implementation does not support changing or querying this parameter. Unless
            otherwise specified, it {\bf MUST} be hardcoded to the default.
        \end{itemize}
    \end{itemize}
\end{itemize}

\subsubsection{\texttt{0x66 `f'} -- FLOW (GOC) interface message}
{\em Synchronous Request}
\begin{itemize}
  \item FLOW messages are formatted the exact same as `d'iscrete messages.
\end{itemize}

\subsubsection{\texttt{0x4f `O'} -- Query optical (FLOW (GOC)) Configuration}
{\em Synchronous Request}
\begin{itemize}
  \item[]
    \begin{bytefield} \\
      \colorbitbox{lightgreen}{8}{`O' (0x4f)} &
      \colorbitbox{lightred}{8}{Event ID} &
      \colorbitbox{lightcyan}{8}{Len (Must be 1)} &
      \bitbox{8}{Parameter} &
    \end{bytefield}
  \item These messages complement the set {\tt `o'} messages.
  \item The `Parameter' field is the parameter specifier to query.
  \item An ACK response should mimic the corresponding set message.
\end{itemize}

\subsubsection{\texttt{0x6f `o'} -- Set optical (FLOW (GOC)) Configuration}
{\em Synchronous Request}
\begin{itemize}
  \item The first byte of the message shall define which parameter is to
    be configured.
    \paragraph{\texttt{0x6f 0x63 `oc'}: Clock Speed (Divider)}
      \begin{itemize}
        \item[]
          \begin{bytefield} \\
            \colorbitbox{lightgreen}{8}{`o'} &
            \colorbitbox{lightred}{8}{Event ID} &
            \colorbitbox{lightcyan}{8}{Len (Must be 4)} &
            \colorbitbox{lightergreen}{8}{`c'} &
            \bitbox{8}{Clock Divider}
          \end{bytefield}
        \item {\bf Default:} {\tt 0x30D400} (2~MHz / 0x30D400 = .625~Hz)
        \item This shall be followed by a three byte
          value N (MSB-first), where 2~MHz / N yields the desired clock speed.
      \end{itemize}
    \paragraph{\texttt{0x6f 0x6f `oo'}: Light On/Off}
      \begin{itemize}
        \item[]
          \begin{bytefield} \\
            \colorbitbox{lightgreen}{8}{`o'} &
            \colorbitbox{lightred}{8}{Event ID} &
            \colorbitbox{lightcyan}{8}{Len (Must be 2)} &
            \colorbitbox{lightergreen}{8}{`o'} &
            \bitbox{8}{On / Off}
          \end{bytefield}
        \item {\bf Default:} {\tt 0x0} (Off)
        \item This shall be followed by either a {\tt 0} or a {\tt 1}
          indicating whether the default state of the light should be on or
          off. If the default is set to on, the light shall remain illuminated
          until set to off or a GOC pulse temporarily turns if off.
      \end{itemize}
  \paragraph{\texttt{0x6f} Responses}
    \begin{itemize}
      \item NAKs for this message shall be composed of an error
        code, followed by an optional explanitory string.
        \begin{itemize}
          \item[]
            \begin{bytefield} \\
              \colorbitbox{lightgreen}{8}{NAK (0x01)} &
              \colorbitbox{lightred}{8}{Event ID} &
              \colorbitbox{lightcyan}{8}{Len (Min: 1)} &
              \colorbitbox{lightblue}{8}{EINVAL (0x16)} &
              \bitbox{8}{[{\tt "Out of Range"}]} &
            \end{bytefield}
          \item {\texttt {\textbf EINVAL (22,0x16):}} Invalid argument.
        \end{itemize}
    \end{itemize}
\end{itemize}

\subsubsection{\texttt{0x42 `B'} -- MBus Snooped Message}
{\em Asynchronous Message}
\begin{itemize}
    % As always, feel free to modify any of this to be easier to match your
    % current implementation.
  \item MBus messages are formatted the exact same as `d'iscrete messages.
  \item This message is sent only from the ICE board to report messages it has
    snooped (but not ACK'd).
  \item Messages that match both the ICE address and the snoop address (that
    is, would generate both {\tt `b'} and {\tt `B'} messages) shall not report
    snoop messages.
  \item The ICE board shall not \hl{(or shall if it's easier, I really don't
    care)} report snoop messages for message that it is sending.
  \item At the end of the message, snoop messages shall append one additional
    byte. This additional byte shall indicate the control bits of the snooped
    message. Control Bit~0 shall be mapped to bit \hl{7 or 0} and Control
    Bit~1 shall be mapped to bit \hl{6 or 1}. The remaining bits are undefined.
\end{itemize}

\subsubsection{\texttt{0x62 `b'} -- MBus Message}
{\em Synchronous Request, Asynchronous Message}
\begin{itemize}
  \item MBus messages are formatted the exact same as `d'iscrete messages.
  \item This command may be sent to the ICE board to send a message.
  \item This message is sent from the ICE board to report messages sent to
    ICE. That is messages that matched the address masks assigned to ICE and
    that ICE has ACK'd on the bus.
\end{itemize}

\subsubsection{\texttt{0x4d `M'} -- Query MBus Configuration}
{\em Synchronous Request}
\begin{itemize}
  \item[]
    \begin{bytefield} \\
      \colorbitbox{lightgreen}{8}{`M' (0x4d)} &
      \colorbitbox{lightred}{8}{Event ID} &
      \colorbitbox{lightcyan}{8}{Length} &
      \bitbox{8}{Parameter} &
    \end{bytefield}
  \item These messages complement the set of {\tt `m'} messages.
  \item The `Parameter' field is the parameter specifier to query.
  \item An ACK response should mimic the corresponding set message.
\end{itemize}

\subsubsection{\texttt{0x6d `m'} -- Set MBus Configuration}
{\em Synchronous Request}
\begin{itemize}
  \item The first byte of the message shall define which parameter is to
    be configured.
    \paragraph{\texttt{0x6d 0x6c `ml'}: Set MBus Full Prefix}
      \begin{itemize}
        \item[]
          \begin{bytefield} \\
            \colorbitbox{lightgreen}{8}{`m'} &
            \colorbitbox{lightred}{8}{Event ID} &
            \colorbitbox{lightcyan}{8}{Len (Must be 7)} &
            \colorbitbox{lightergreen}{8}{`l'} &
            \bitbox{8}{Ones Mask} &
            \bitbox{8}{Zeros Mask}
          \end{bytefield}
        \item {\bf Default:} {\tt 0xfffff0 0xfffff0} (Disabled)
        \item This shall be followed by 6 bytes. The first three bytes shall
          be considered the ones mask the second three bytes shall be
          considered the zeros mask. The masks are 20~bits long, the bottom
          4~bits of the transmitted masks shall be ignored.
      \end{itemize}
    \paragraph{\texttt{0x6d 0x73 `ms'}: Set MBus Short Prefix}
      \begin{itemize}
        \item[]
          \begin{bytefield} \\
            \colorbitbox{lightgreen}{8}{`m'} &
            \colorbitbox{lightred}{8}{Event ID} &
            \colorbitbox{lightcyan}{8}{Len (Must be 3)} &
            \colorbitbox{lightergreen}{8}{`s'} &
            \bitbox{8}{Ones Mask} &
            \bitbox{8}{Zeros Mask}
          \end{bytefield}
        \item {\bf Default:} {\tt 0xf0 0xf0} (Disabled)
        \item This shall be followed by 2 bytes. The first byte shall
          be considered the ones mask the second byte shall be
          considered the zeros mask. The masks are 4~bits long, the bottom
          4~bits of the transmitted masks shall be ignored.
        \item {\bf NOTE:} Changing the short prefix after enumeration is a
          violation of the MBus protocol. The ICE board will permit this, but
          it may have unexpected consequences.
      \end{itemize}
    \paragraph{\texttt{0x6d 0x4c `mL'}: Set MBus Snoop Full Prefix}
      \begin{itemize}
        \item[]
          \begin{bytefield} \\
            \colorbitbox{lightgreen}{8}{`m'} &
            \colorbitbox{lightred}{8}{Event ID} &
            \colorbitbox{lightcyan}{8}{Len (Must be 7)} &
            \colorbitbox{lightergreen}{8}{`L'} &
            \bitbox{8}{Ones Mask} &
            \bitbox{8}{Zeros Mask}
          \end{bytefield}
        \item {\bf Default:} {\tt 0xfffff0 0xfffff0} (Disabled)
        \item This shall be followed by 6 bytes. The first three bytes shall
          be considered the ones mask the second three bytes shall be
          considered the zeros mask. The masks are 20~bits long, the bottom
          4~bits of the transmitted masks shall be ignored.
        \item Messages matching the snoop prefix shall be reported by a
          {\tt `B'} message but will not be ACK'd on the physical bus by ICE.
        \item In the case of a conflict between a snoop and a match, the match
          takes precedence and ICE will ACK the message.
      \end{itemize}
    \paragraph{\texttt{0x6d 0x53 `mS'}: Set MBus Snoop Short Prefix}
      \begin{itemize}
        \item[]
          \begin{bytefield} \\
            \colorbitbox{lightgreen}{8}{`m'} &
            \colorbitbox{lightred}{8}{Event ID} &
            \colorbitbox{lightcyan}{8}{Len (Must be 3)} &
            \colorbitbox{lightergreen}{8}{`S'} &
            \bitbox{8}{Ones Mask} &
            \bitbox{8}{Zeros Mask}
          \end{bytefield}
        \item {\bf Default:} {\tt 0xf0 0xf0} (Disabled)
        \item This shall be followed by 2 bytes. The first byte shall
          be considered the ones mask the second byte shall be
          considered the zeros mask. The masks are 4~bits long, the bottom
          4~bits of the transmitted masks shall be ignored.
        \item Messages matching the snoop prefix shall be reported by a
          {\tt `B'} message but will not be ACK'd on the physical bus by ICE.
        \item In the case of a conflict between a snoop and a match, the match
          takes precedence and ICE will ACK the message.
      \end{itemize}
    \paragraph{\texttt{0x6d 0x73 `mb'}: Set MBus Broadcast Mask}
      \begin{itemize}
        \item[]
          \begin{bytefield} \\
            \colorbitbox{lightgreen}{8}{`m'} &
            \colorbitbox{lightred}{8}{Event ID} &
            \colorbitbox{lightcyan}{8}{Len (Must be 1)} &
            \colorbitbox{lightergreen}{8}{`b'} &
            \bitbox{8}{Ones Mask} &
            \bitbox{8}{Zeros Mask}
          \end{bytefield}
        \item {\bf Default:} {\tt 0x0f 0x0f} (Disabled)
        \item This shall be followed by 2 bytes. The first byte shall
          be considered the ones mask the second byte shall be
          considered the zeros mask. The masks are 4~bits long, the top
          4~bits of the transmitted masks shall be ignored.
        \item These masks are the set of broadcast channels that the ICE board
          should ACK on the physical MBus.
      \end{itemize}
    \paragraph{\texttt{0x6d 0x53 `mB'}: Set MBus Snoop Broadcast Mask}
      \begin{itemize}
        \item[]
          \begin{bytefield} \\
            \colorbitbox{lightgreen}{8}{`m'} &
            \colorbitbox{lightred}{8}{Event ID} &
            \colorbitbox{lightcyan}{8}{Len (Must be 1)} &
            \colorbitbox{lightergreen}{8}{`B'} &
            \bitbox{8}{Ones Mask} &
            \bitbox{8}{Zeros Mask}
          \end{bytefield}
        \item {\bf Default:} {\tt 0x0f 0x0f} (Disabled)
        \item This shall be followed by 2 bytes. The first byte shall
          be considered the ones mask the second byte shall be
          considered the zeros mask. The masks are 4~bits long, the top
          4~bits of the transmitted masks shall be ignored.
        \item These masks are the set of broadcast channels that the ICE board
          will not ACK on the physical MBus but will report as a snooped
          message (a {\tt `B'} message).
        \item In the case of a conflict between a snoop and a match, the match
          takes precedence and ICE will ACK the message.
      \end{itemize}
    \paragraph{\texttt{0x6d 0x6d `mm'}: Set master mode on/off}
      \begin{itemize}
        \item[]
          \begin{bytefield} \\
            \colorbitbox{lightgreen}{8}{`m'} &
            \colorbitbox{lightred}{8}{Event ID} &
            \colorbitbox{lightcyan}{8}{Len (Must be 2)} &
            \colorbitbox{lightergreen}{8}{`m'} &
            \bitbox{8}{\texttt{0x00} or \texttt{0x01}}
          \end{bytefield}
        \item {\bf Default:} {\tt 0x0} (Off)
        \item Boolean whether ICE should act as the MBus master node (1 - on)
          or as a MBus member node (0 - off).
      \end{itemize}
    \paragraph{\texttt{0x6d 0x63 `mc'}: Clock Speed}
      \begin{itemize}
        \item[]
          \begin{bytefield} \\
            \colorbitbox{lightgreen}{8}{`o'} &
            \colorbitbox{lightred}{8}{Event ID} &
            \colorbitbox{lightcyan}{8}{Len (??)} &
            \colorbitbox{lightergreen}{8}{`c'} &
            \bitbox{8}{???}
          \end{bytefield}
        \item {\bf Default:} \hl{XXX}
        \item \hl{XXX: However you would like to specify this in HW. Something
            that gives a pretty wide range (say 1~kHz -- 10~MHz) would be
          good. Precision is not as important.}
        \item This configuration is meaningful only if ICE is configured as
          the MBus master node (see {\tt `mm'}).
      \end{itemize}
    \paragraph{\texttt{0x6d 0x69 `mi'}: Set should interrupt}
      \begin{itemize}
        \item[]
          \begin{bytefield} \\
            \colorbitbox{lightgreen}{8}{`m'} &
            \colorbitbox{lightred}{8}{Event ID} &
            \colorbitbox{lightcyan}{8}{Len (Must be 2)} &
            \colorbitbox{lightergreen}{8}{`i'} &
            \bitbox{8}{{\tt SHOULD\_INT}}
          \end{bytefield}
        \item {\bf Default:} {\tt 0x0} (Off)
        \item Configures whether ICE should interrupt the bus to transmit its
          next message. Possible values:
          \begin{itemize}
            \item[0] Do not interrupt
            \item[1] Interrupt (if necessary) for next message only. That is,
              this value is reset to 0 after the next {\tt `b'} command sent
              to ICE is handled.
            \item[2] Interrupt for all messages
          \end{itemize}
      \end{itemize}
    \paragraph{\texttt{0x6d 0x69 `mp'}: Set should use priority arbitration}
      \begin{itemize}
        \item[]
          \begin{bytefield} \\
            \colorbitbox{lightgreen}{8}{`m'} &
            \colorbitbox{lightred}{8}{Event ID} &
            \colorbitbox{lightcyan}{8}{Len (Must be 2)} &
            \colorbitbox{lightergreen}{8}{`p'} &
            \bitbox{8}{{\tt SHOULD\_PRIO}}
          \end{bytefield}
        \item {\bf Default:} {\tt 0x0} (Off)
        \item Configures whether ICE should send a high priority message for
          the next message. Possible values:
          \begin{itemize}
            \item[0] Do not send priority message.
            \item[1] Send priority for next message only. That is, this value
              is reset to 0 after the next {\tt `b'} command sent to ICE is
              handled.
            \item[2] Send all messages as high priority.
          \end{itemize}
      \end{itemize}
  \paragraph{\texttt{0x6d} Responses}
    \begin{itemize}
      \item NAKs for this message shall be composed of an error
        code, followed by an optional explanitory string.
        \begin{itemize}
          \item[]
            \begin{bytefield} \\
              \colorbitbox{lightgreen}{8}{NAK (0x01)} &
              \colorbitbox{lightred}{8}{Event ID} &
              \colorbitbox{lightcyan}{8}{Len (Min: 1)} &
              \colorbitbox{lightblue}{8}{EINVAL (0x16)} &
              \bitbox{8}{[{\tt "Out of Range"}]} &
            \end{bytefield}
          \item {\texttt {\textbf EINVAL (22,0x16):}} Invalid argument.
          \item[]
            \begin{bytefield} \\
              \colorbitbox{lightgreen}{8}{NAK (0x01)} &
              \colorbitbox{lightred}{8}{Event ID} &
              \colorbitbox{lightcyan}{8}{Len (Min: 1)} &
              \colorbitbox{lightblue}{8}{ENODEV (0x13)} &
            \end{bytefield}
          \item {\texttt {\textbf ENODEV (19,0x13):}} The
            implementation does not support changing or querying this parameter. Unless
            otherwise specified, it {\bf MUST} be hardcoded to the default.
        \end{itemize}
    \end{itemize}
\end{itemize}

\subsubsection{\texttt{0x65 `e'} -- \texttt{EIN} debug port message}
{\em Synchronous Request}
\begin{itemize}
  \item \texttt{EIN} debug port messages are formatted the exact same as
    `d'iscrete messages.
\end{itemize}

\subsubsection{\texttt{0x47 `G'} -- Query GPIO State / Configuration}
{\em Synchronous Request}
\begin{itemize}
  \item[]
    \begin{bytefield} \\
      \colorbitbox{lightgreen}{8}{`G' (0x47)} &
      \colorbitbox{lightred}{8}{Event ID} &
      \colorbitbox{lightcyan}{8}{Len (Must be 2)} &
      \colorbitbox{lightergreen}{8}{Parameter} &
      \bitbox{8}{GPIO IDX}
    \end{bytefield}
  \item These messages complement the set GPIO (`g') messages.
  \item The `Parameter' field is the parameter specifier to query.
  \item An ACK response should mimic the corresponding set message.
\end{itemize}

\subsubsection{\texttt{0x67 `g'} -- Set / Configure GPIO}
{\em Synchronous Request, Asynchronous Message}
\begin{itemize}
  \item The first byte of this message shall be a specificer, indicating
    what type of GPIO action is requested.
  \paragraph{\texttt{0x67 0x6c `gl'} -- GPIO Level}
    \begin{itemize}
      \item The first byte of the message shall be an integer
        indicating the GPIO index to set. The second byte shall be
        valued {\tt 0} or {\tt 1}, depending on the desired GPIO
        state.
      \item[]
        \begin{bytefield} \\
          \colorbitbox{lightgreen}{8}{`o' (0x67)} &
          \colorbitbox{lightred}{8}{Event ID} &
          \colorbitbox{lightcyan}{8}{Len (Must be 3)} &
          \colorbitbox{lightergreen}{8}{`l' (0x6c)} &
          \bitbox{8}{GPIO IDX} &
          \bitbox{8}{GPIO Val}
        \end{bytefield}
    \end{itemize}
  \paragraph{\texttt{0x67 0x64 `gd'} -- GPIO direction}
    \begin{itemize}
      \item The first byte of the message shall be an integer
        indicating the GPIO index to set the direction of. The second
        byte shall be valued:
        \begin{itemize}
          \item 0: Input
          \item 1: Output
          \item 2: TriState (\textsc{DEFAULT})
        \end{itemize}
      \item[]
        \begin{bytefield} \\
          \colorbitbox{lightgreen}{8}{`o' (0x67)} &
          \colorbitbox{lightred}{8}{Event ID} &
          \colorbitbox{lightcyan}{8}{Len (Must be 3))} &
          \colorbitbox{lightergreen}{8}{`d' (0x64)} &
          \bitbox{8}{GPIO IDX} &
          \bitbox{8}{GPIO Direction}
        \end{bytefield}
    \end{itemize}
  \paragraph{\texttt{0x67} Responses}
    \begin{itemize}
      \item {\texttt {\textbf ENODEV (19,0x13):}} The requested
        GPIO does not exist.
      \item[]
        \begin{bytefield} \\
          \colorbitbox{lightgreen}{8}{NAK (0x01)} &
          \colorbitbox{lightred}{8}{Event ID} &
          \colorbitbox{lightcyan}{8}{Len (Min: 1)} &
          \colorbitbox{lightblue}{8}{ENODEV (0x13)} &
          \bitbox{8}{[{\tt "No such GPIO"}]} &
        \end{bytefield}
      \item {\texttt {\textbf EINVAL (22,0x16):}} The requested GPIO
        exists, but cannot be configured this way at this time.
      \item[]
        \begin{bytefield} \\
          \colorbitbox{lightgreen}{8}{NAK (0x01)} &
          \colorbitbox{lightred}{8}{Event ID} &
          \colorbitbox{lightcyan}{8}{Len (Min: 1)} &
          \colorbitbox{lightblue}{8}{EINVAL (0x16)} &
          \bitbox{8}{[{\tt "GPIO is input"}]} &
        \end{bytefield}
    \end{itemize}
\end{itemize}

\subsubsection{\texttt{0x50 `P'} -- Query Power State}
{\em Synchronous Request}
\begin{itemize}
  \item[]
    \begin{bytefield} \\
      \colorbitbox{lightgreen}{8}{`P' (0x50)} &
      \colorbitbox{lightred}{8}{Event ID} &
      \colorbitbox{lightcyan}{8}{Len (Must be 1)} &
      \colorbitbox{lightergreen}{8}{Parameter} &
      \bitbox{8}{PWR IDX}
    \end{bytefield}
  \item These messages complement the Set Power (`p') messages.
  \item The `Parameter' field is the parameter specifier to query.
  \item An ACK response should mimic the corresponding set message.
\end{itemize}

\subsubsection{\texttt{0x70 `p'} -- Set Power State}
{\em Synchronous Request, Asynchronous Message}
\begin{itemize}
  \item Set Power State messages allow direct control of set-point voltage and on/off states for various power domains on the ICE board.  The first byte of this message shall be a specificer, indicating which parameter is requested.  The second byte of the message shall be the power domain identifier.  Currently implemented power domain identifiers are:
   \begin{itemize}
     \item 0: M3 0.6V (0.675V Default)
     \item 1: M3 1.2V (1.2V Default)
     \item 2: M3 VBatt (3.8V Default)
   \end{itemize}
  \paragraph{\texttt{0x70 0x76 `pv'} -- Voltage State}
    \begin{itemize}
      \item The first byte of the message shall be a single byte indicating the power domain identifier to set.  The second byte ($v\_set$) shall specify the voltage according to the equation:
$$V_{out} = (0.537 + 0.0185*v\_set)*V_{default}$$
      Valid values for $v\_set$ range from {\tt 0} to {\tt 31}
      \item[]
        \begin{bytefield} \\
          \colorbitbox{lightgreen}{8}{`p' (0x70)} &
          \colorbitbox{lightred}{8}{Event ID} &
          \colorbitbox{lightcyan}{8}{Len (Must be 3)} &
          \colorbitbox{lightergreen}{8}{`v' (0x76)} &
          \bitbox{8}{PWR IDX} &
          \bitbox{8}{$v\_set$}
        \end{bytefield}
    \end{itemize}
  \paragraph{\texttt{0x70 0x6f `po'} -- On/Off State}
    \begin{itemize}
      \item The first byte of the message shall be a single byte indicating the power domain identifier to set.  The second byte shall be
        valued {\tt 0} or {\tt 1}, depending on the desired On/Off
        state.
      \item[]
        \begin{bytefield} \\
          \colorbitbox{lightgreen}{8}{`p' (0x70)} &
          \colorbitbox{lightred}{8}{Event ID} &
          \colorbitbox{lightcyan}{8}{Len (Must be 3)} &
          \colorbitbox{lightergreen}{8}{`o' (0x6f)} &
          \bitbox{8}{PWR IDX} &
          \bitbox{8}{On/Off}
        \end{bytefield}
    \end{itemize}
\end{itemize}


%%%%%%%%%%%%%%%%%%%%%%%%%%%%%%%%%%%%%%%%%%%%%%%%%%%%%%%
\pagebreak
\section{Document Revision History}
\label{sec:revisions}

\begin{itemize}

\item Revision 0.2.7 -- Oct 29, 2013
\subitem Add `?c' message

\item Revision 0.2.6 -- Oct 17, 2013
\subitem Add `Oo' and `oo' messages

\item Revision 0.2.5 -- Jan 14, 2013
\subitem Add `p' and `P' messages

\item Revision 0.2.4 -- Dec 20, 2012
\subitem Add `g' and `G' messages

\item Revision 0.2.3 -- Dec 18, 2012
\subitem Add `f', `o', and `O' messages

\item Revision 0.2.2 -- Dec 15, 2012
\subitem `d' fragments are 255 bytes long so that a final fragment of maximum
length can be distinguished
\subitem Make explicit that each `d' fragment message is ACK/NAK'd
\subitem When a `d' fragment is NAK'd, that ends the transaction

\item Revision 0.2.1 -- Nov 29, 2012
\subitem 0.1's `d' $\rightarrow$ `e'
\subitem Add fragmented `d' messages

\item Revision 0.2 -- Nov 27, 2012
\subitem Largely redefine the protocol (ish)

\item Revision 0.1.2 -- Nov 26, 2012
\subitem Restrict Version to `V' only
\subitem Restrict eXtension to `X' only

\item Revision 0.1.1 -- Nov 16, 2012
\subitem Add visualized packets

\item Revision 0.1 -- Nov 14, 2012
\subitem Initial revision

\end{itemize}

\end{document}
